\chapter{Conclusions and future directions}
\label{chapter7}

The present chapter serves the purpose of reviewing and comparing the results obtained in the previous chapter for the analytical, the ILP and the heuristic models for the different networks and traffic scenarios designed, in order to elaborate some master conclusions. This chapter is divided in two main sections, in section \ref{conclusions} is made a review on the work developed along the dissertation and summarize the main conclusions obtained and in section \ref{future} some suggestions for improvement and future research are provided.


\section{Conclusions}
\label{conclusions}

This dissertation begins by introducing some of the main concepts on the optical networking area and a more specific explanation on transparent networks. In order to calculate the CAPEX of a network different models were used, more specifically, an analytical and a heuristic model here proposed and an ILP from a previous dissertation was used for comparison purposes. To perform a detailed study of this models two different networks were used, a more simplistic reference network, composed of 6 nodes and 8 bidirectional links, and one other real and more complex network comprised of 12 nodes and 17 bidirectional links. Additionally, various traffic scenarios were defined for each network in order to analyze their behavior. Throughout this dissertation it was developed and implemented a set of heuristic algorithms, for transparent optical networks dimensioning purposes.  In order to implement those algorithms a generic framework had to be created and then implemented over the NetXPTO-NetPlanner simulator, an academic open source real-time simulator, which allows the creation of generic systems comprised of a set of blocks that interact with each other through signals. The created simulator is extremely adaptable since it contemplates a vast list of entry parameters, representing the specific characteristics of the network to be tested, that may be modified according to the user preferences. Although the simulator only contemplates the case of transparent networks without survivability this platform was designed to be sufficiently generic so that in the future it can be complemented with other features such as different transport modes (opaque and translucent) and different protection schemes, such as, 1+1 and dedicated paths. A final report is also generated after the completion of each simulation and the model of costs, provided in this dissertation, is applied in order to obtain the heuristics results presented in chapter \ref{chapter6}. The objective was to perform two comparative studies regarding the CAPEX solutions provided and the time of execution metrics of the different models used. This study was initially performed for the reference network in order to evaluate the existent gap between the CAPEX values provided by the ILP model, considered to be the optimal solutions, and the ones given by the developed heuristics and the analytical model just for a matter of calibration. Looking now for the results obtained we concluded that the heuristics applied were well calibrated since the margin of error towards the ILP results was minimal, under 2\% for the medium and high traffic scenarios, or even nonexistent for the low traffic scenario, which means that for simpler cases the heuristics performance can match the ILPs. Having that said, it was now possible to apply the heuristics to a real and more complex network, in this case the chosen network was the vBNS, where a medium-high traffic scenario was applied. Here only the heuristic algorithms were capable of providing dimensioning solution in a reasonable amount of time, about 16 seconds, while the ILP model was not capable of encountering the solution under a maximum defined timeout period of 24 hours, thus, once again showing that these algorithms may be a good solution for real and more complex problems considering that the ILP models for these cases take a long time to obtain results.


\section{Future directions}
\label{future}

During this dissertation some specific situations were analyzed and some open issues were discovered. Since there is always space for improvement especially when optimization is involved, some future work suggestions are provided below:
\begin{enumerate}
    \item Allow the possibility of existing protected traffic, the present platform does not take into account either the existence of shared protection or dedicated link protection.
    \item Implement the opaque and translucent transport modes, since this platform only considers the possibility of transparent networks.
    \item Perform studies considering multiple transmission systems in each link, although the present platform allows this possibility, this case was not studied.
    \item Regarding the scheduling algorithm, consider the use of other metrics, such as, the length of the shortest logical or physical paths in terms of distance, the need for protection paths, the quality of the path set in terms of desirable path options or a smart combination of some of this aspects.
    \item Regarding the routing strategy, consider the use of other metrics, based for example on the lengths of the shortest paths and the need for protection or optical regeneration.
    \item Giving continuity to the GIT repository, where all the documents and developed code regarding this dissertation and some other previous works in the same field of investigation, were released.
\end{enumerate}

\cleardoublepage